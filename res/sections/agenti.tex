\newpage

\section{Cos'è l'IA?}

Ci sono 4 visioni dell'intelligenza artificiale: è qualcosa che deve\dots

\begin{itemize}
 \item Pensare come l'essere umano?
 \item Pensare razionalmente?
 \item Agire come l'essere umano?
 \item Agire razionalmente?
\end{itemize}

Questi punti di vista derivano dal \textbf{test di Turing}, in cui ci si
interroga sul confine e sulle differenze tra ciò che può fare una macchina
e un essere umano.

Il test di Turing è un gioco che coinvolge tre persone: un uomo (A), una
donna (B) e un terzo individuo (C).

C deve indovinare, ponendo una serie di domande, il sesso di A e B.

A dovrà cercare di ingannare C, B dovrà cercare di aiutarlo a risolvere il
quesito.

Le risposte alle domande dovranno essere dattiloscritte, per evitare
che la grafia e la voce possano aiutare C a trovare la soluzione.

Turing ipotizza che alla persona A si sostituisca una macchina.

Se C dopo questa sostituzione non si accorge di nulla allora A dovrebbe
essere considerata intelligente quanto un essere umano.

\subsection{Macro aree dell'IA}

Le macro aree che vengono individuate grazie al test di Turing sono:

\begin{itemize}
 \item Natural language understanding/processing
 \item Knowledge representation
 \item Automatic reasoning
 \item Experential learning
 \item Computer vision e robotics (total Turing test)
\end{itemize}

Per ogni classe di ambienti e attività viene scelto il miglior agente (l'agente
con le migliori performances).

\subsection{Agenti intelligenti}

Idealmente, un agente intelligente dovrebbe prendere la migliore decisione
possibile in ogni situazione.

Un agente \textbf{percepisce} l'ambiente con dei sensori e \textbf{agisce}
attraverso degli attuatori.\\

\textbf{Funzione agente f}:

\begin{equation}
f: P^* \rightarrow A
\end{equation}

Il programma agente gira su un'architettura fisica per produrre la funzione f.

\textbf{Il programma agente è diverso dalla funzione agente}.
Il primo considera la percezione corrente, la seconda considera tutta la
storia percettiva ($P^*$).

Un agente si dice \textbf{autonomo} se impara ad adattarsi all'ambiente.

\subsection{Programma agente}

Un programma agente consiste di più componenti, che possono rappresentare
l'ambiente in tre modi diversi:

\begin{itemize}
 \item \textbf{Atomic}: ogni stato del mondo è indivisibile (non ha una
struttura interna).
 \item \textbf{Factored}: ogni stato ha un insieme fissato di variabili o
attributi (che hanno dei valori associati).
Due stati separati possono condividere le stesse variabili.
 \item \textbf{Structured}: oggetti con relazioni tra loro, non solo
variabili.
\end{itemize}
