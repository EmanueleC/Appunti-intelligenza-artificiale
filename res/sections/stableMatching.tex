\newpage

\section{Stable Matching Problem}


Il problema è quello di trovare dei match tra gli agenti di due gruppi
in base alle loro preferenze. I match devono essere \textbf{stabili}, cioè
non devono esistere due agenti che si preferirebbero tra di loro piuttosto
che con il rispettivo partner.

Ad esempio a un insieme di rovers \{Rover1, Rover2, Rover3\} sono associate
delle preferenze rispetto a un insieme di analisi \{Drill, Picture, Download\}
e viceversa.

Le preferenze dei rovers sono:

\begin{itemize}
 \item Rover1:  Downlink > Picture > Drill
 \item Rover2:  Picture > Downlink > Drill
 \item Rover3:  Downlink > Picture > Drill
\end{itemize}

Le preferenze delle analisi sono:

\begin{itemize}
 \item Downlink: Rover2 > Rover1 > Rover3
 \item Picture: Rover1 > Rover2 > Rover3
 \item Drill: Rover1 > Rover2 > Rover3
\end{itemize}

L'obiettivo è trovare dei match stabili tali che ogni rover è assegnato a
un'analisi e non esiste nessuna coppia (rover, analisi) che vorrebbe rompersi
e formare un nuovo match.

L'algoritmo \ref{alg:galshap} di Gale Shapley raggiunge questo obiettivo.

\begin{algorithm}
    \caption{Algoritmo di Gale Shapley}
    \label{alg:galshap}
    \begin{algorithmic}[1] % The number tells where the line numbering should start
        \Procedure{STABLE MATCHING}{}
            \While{esiste un uomo single u}
            \State u si propone alla donna d che preferisce con cui non ci
ha già provato
            \If{d è single} \State{(u,d) diventano una coppia}
            \Else
            \If{d preferisce u al suo partner p} \State{(u,d) diventano una
coppia e p diventa libero}
            \Else \State{d rifiuta la proposta} \EndIf
            \EndIf
            \EndWhile
        \EndProcedure
    \end{algorithmic}
\end{algorithm}

Ciascuno degli n agenti può fare al più n proposte: la complessità
computazionale di \ref{alg:galshap} è $\mathcal{O}(n^2)$.\\

Può esistere più di un matching stabile. L'algoritmo \ref{alg:galshap}
ritorna una \textbf{soluzione che è ottima per gli uomini, ma la peggiore per
le donne} (significa che in qualunque altro matching stabile le donne sono
più soddisfatte).

Si applica l'algoritmo all'esempio dei rovers e delle analisi: i
rovers sono gli uomini, le analisi le donne\dots in questo modo si troverà
una soluzione ottima per i rovers.\\

All'inizio tutti i rovers e tutte le analisi sono single.

Rover1 si propone a Downlink che è libera $\implies$ (Rover1, Downlink).

Rover2 si propone a Picture che è libera $\implies$ (Rover2, Picture).

Rover3 si propone a Downlink, ma Downlink è già occupata con Rover1 e non
preferisce Rover3 a Rover1.

Rover3 si propone alla sua seconda preferenza: Picture, ma Picture è già
occupata con Rover2 e non preferisce Rover3 a Rover2.

Rover3 si propone alla sua terza preferenza: Drill, che è libera $\implies$
(Rover3, Drill).\\

Il matching stabile trovato dall'algoritmo è \textbf{\{(Rover1, Downlink),
(Rover2, Picture), (Rover3, Drill)\}}.

\subsection{Preferenze a pari merito}

Ci possono essere degli uomini che non preferiscono una donna a un'altra,
ciò significa che esistono preferenze a pari merito.

Con questa osservazione, il concetto di stabilità si amplia: un matching
può essere\dots

\begin{itemize}
 \item \textbf{fortemente stabile} non esiste una coppia di persone singles
tale che uno dei due preferisce strettamente l'altro e l'altro lo preferisce
equalmente o di più.
 \item \textbf{debolmente stabile} non esiste una coppia di persone singles
che preferiscono strettamente l'un l'altro.
\end{itemize}

Un matching fortemente stabile potrebbe non esistere. Per controllare la
sua esistenza occorre complessità $\mathcal{O}(n^4)$. Al contrario, un
matching debolmente stabile esiste sempre. Basta solo rompere le preferenze
a pari merito in modo arbitrario e usare l'algoritmo di Gale Shapley.

\subsection{Esercizio}

Apply Gale-Shapley algorithm to the preference profile shown below.
What is the resulting matching?
What is the property of the resulting matching?

Preferences for men:

\begin{itemize}
 \item Adam: Cindy > Allie > Betty
 \item Blake: Betty > Allie > Cindy
 \item Carl: Cindy > Betty > Allie
\end{itemize}

Preferences for women:

\begin{itemize}
 \item Allie: Adam > Blake > Carl
 \item Betty: Carl > Blake > Adam
 \item Cindy: Adam > Carl > Blake
\end{itemize}

Start

\{\}

\{(Adam, Cindy)\}

\{(Adam, Cindy), (Blake, Betty)\}

\{(Adam, Cindy), (Carl, Betty)\}

\textbf{\{(Adam, Cindy), (Carl, Betty), (Blake, Allie)\}}

End

The resulting matching is complete and stable: it means that there are no 2
people that would prefer each other over the respective partner.



