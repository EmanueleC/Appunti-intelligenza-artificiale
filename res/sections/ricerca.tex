\newpage

\section{Ricerca}

Un \textbf{albero di ricerca} è dato dall'insieme delle possibili sequenze di
azioni che è possibile applicare a partire da uno stato iniziale.

\textbf{Radice}: stato iniziale.

\textbf{Nodi}: stati.

\textbf{Rami}: azioni possibili.

Lo stesso stato può comparire più volte $\rightarrow$ possono esserci dei cicli.

\textbf{Nota bene}: il grafo che rappresenta lo spazio degli stati è una cosa
diversa dall'albero di ricerca.

\textbf{Foglie}: nodi senza figli.

\textbf{Frontiera}: nodi foglia che si possono espandere.

\lstinputlisting[language=Python]{source_filename.py}
function TREE-SEARCH(problem) returns a solution, or failure
// initialize the frontier using the initial state of problem
loop do
  if the frontier is empty then return failure
  choose a leaf node and remove it from the frontier
  if the node contains a goal state, returns the solution
  expand the chosen node, adding the resulting nodes to the frontier







