\newpage

\section{Ricerca}

Un \textbf{albero di ricerca} è dato dall'insieme delle possibili sequenze di
azioni che è possibile applicare a partire da uno stato iniziale.

\textbf{Radice}: stato iniziale.

\textbf{Nodi}: stati.

\textbf{Rami}: azioni possibili.

Lo stesso stato può comparire più volte $\rightarrow$ possono esserci dei cicli.

\textbf{Nota bene}: il grafo che rappresenta lo spazio degli stati \textbf{è una
cosa diversa dall'albero di ricerca}.

\textbf{Foglie}: nodi senza figli.

\textbf{Frontiera}: nodi foglia che si possono espandere.

\begin{algorithm}
    \caption{Algoritmo di ricerca}
    \label{search}
    \begin{algorithmic}[1] % The number tells where the line numbering should start
        \Procedure{TREE SEARCH}{$problem$} \Comment{ritorna una soluzione o fallisce}
            \Loop
            \If{Frontiera vuota} \Return{Fallimento} \EndIf
            \State Scegli un nodo foglia e rimuovilo dalla frontiera
            \If{Nodo con stato obiettivo} \Return{Soluzione} \EndIf
            \State Espandi il nodo scelto e aggiungi il risultato alla frontiera
			\EndLoop
        \EndProcedure
    \end{algorithmic}
\end{algorithm}






