\newpage

\section{Problemi con vincoli}

CSP (Constraint satisfaction problems) è un modo di risolvere un'ampia varietà di problemi.

Stati definiti da \textbf{ variabili $X_i$ } con valori nel dominio $D_i$

Il \textbf{test goal} è definito da un insieme di vincoli.

La {soluzione} consiste in un insieme di valori (uno per ogni variabile) che
soddisfano tutti i vincoli.

L'idea generale è quella di eliminare larghe porzioni dallo spazio di ricerca
identificando combinazioni di valori/variabili che violano i vincoli.\\

Insieme di vincoli: $C = \{ c_i = (scope,rel_i) | i=1,...,h\}$

Lo \textbf{scope} indica le variabili interessate dal vincolo, mentre \textbf{rel}
indica quali assegnamenti di valori sono permessi.

I vincoli si classificano in:

\begin{itemize}
 \item \textbf{unari} - coinvolgono una variabile
 \item \textbf{binari} - coinvolgono una coppia di variabili
 \item \textbf{higher-order} - coinvolgono 3 o più variabili
 \item \textbf{globali} - coinvolgono un numero arbitrario di variabili
\end{itemize}

Concetti utili:

\textbf{Stato}: un assegnamento di valori a qualcuna o a tutte le variabili.

Un \textbf{Assegnamento} può essere:

\begin{itemize}
 \item \textbf{consistente} se non viola nessun vincolo
 \item \textbf{completo} se ogni variabile viene assegnata a un valore
 \item \textbf{parziale}: non completo
\end{itemize}

\textbf{Soluzione}: un assegnamento consistente e completo.

Un CSP binario contiene solo vincoli binari.

In un grafo di un CSP i nodi rappresentano le variabili, gli archi rappresentano
i vincoli.

Perché tradurre un problema in CSP? Il fatto è che molti problemi presentano
una traduzione intuitiva in un csp. Inoltre esistono sistemi generali in grado di
risolvere problemi csp senza dover costruire soluzioni su misura.
Inoltre i risolutori di CSP permettono di ridurre lo spazio da esplorare grazie
alla propagazione dei vincoli.
